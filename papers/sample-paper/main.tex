\documentclass{article}
\usepackage{amsmath}

\title{Sample LaTeX Paper}
\author{Your Name}
\date{\today}

\begin{document}

\maketitle

\begin{abstract}
This is a sample LaTeX document to demonstrate the functionality of our LaTeX web editor. This paper includes common mathematical notation, sections, and formatting elements typically found in academic papers.
\end{abstract}

\section{Introduction}

This is the introduction section. LaTeX is a high-quality typesetting system that is particularly well-suited for technical and scientific documentation.

\section{Mathematical Equations}

Here are some examples of mathematical notation:

\subsection{Inline Math}
The famous equation $E = mc^2$ demonstrates the relationship between energy and mass.

\subsection{Display Math}
The quadratic formula is:
\begin{equation}
x = \frac{-b \pm \sqrt{b^2 - 4ac}}{2a}
\end{equation}

\subsection{Matrix Example}
A sample matrix:
\begin{equation}
A = \begin{pmatrix}
1 & 2 & 3 \\
4 & 5 & 6 \\
7 & 8 & 9
\end{pmatrix}
\end{equation}

\section{Lists}

\subsection{Itemized List}
\begin{itemize}
    \item First item
    \item Second item
    \item Third item
\end{itemize}

\subsection{Enumerated List}
\begin{enumerate}
    \item First numbered item
    \item Second numbered item
    \item Third numbered item
\end{enumerate}

\section{Conclusion}

This sample document demonstrates various LaTeX features. When you edit this file in your IDE, the web interface should automatically recompile and display the updated PDF.

\end{document}