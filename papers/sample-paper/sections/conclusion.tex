\section{Conclusion}

This paper has demonstrated the multi-file capabilities of our web-based LaTeX editor through practical examples. We showed how complex mathematical expressions can be organized (Equation \ref{eq:quadratic}), how matrices can be presented clearly (Equation \ref{eq:sample-matrix}), and how bibliographic references enhance academic writing \cite{knuth1984literate}.

The key contributions of this work include:
\begin{enumerate}
    \item A comprehensive multi-file LaTeX editing environment
    \item Real-time compilation with error handling
    \item Integrated file tree navigation system
    \item Support for bibliographic management and cross-references
\end{enumerate}

\subsection{Future Work}

Several areas for future development have been identified:
\begin{itemize}
    \item Enhanced syntax highlighting for LaTeX code
    \item Collaborative editing with real-time synchronization
    \item Template library for common document types
    \item Integration with reference management systems like Zotero
\end{itemize}

The system successfully addresses the challenges faced by researchers and students when working with LaTeX documents \cite{web-based-latex}. By providing a structured, web-based environment, we have made LaTeX more accessible while maintaining the power and flexibility that makes it the standard for academic publishing \cite{lamport1994latex}.

In conclusion, this multi-file approach not only improves document organization but also facilitates collaborative research and enhances the overall writing experience for academic authors.