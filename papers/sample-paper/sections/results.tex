\section{Results and Evaluation}

This section presents the results of implementing and testing the multi-file LaTeX editing system.

\subsection{Performance Metrics}

The system was evaluated across several key performance indicators:

\begin{table}[h]
\centering
\begin{tabular}{|l|c|c|}
\hline
\textbf{Metric} & \textbf{Single File} & \textbf{Multi-File} \\
\hline
Compilation Time (avg) & 2.3s & 2.8s \\
File Load Time & 0.1s & 0.3s \\
Memory Usage & 45MB & 52MB \\
Error Detection Rate & 85\% & 92\% \\
\hline
\end{tabular}
\caption{Performance comparison between single-file and multi-file approaches}
\label{tab:performance}
\end{table}

As shown in Table \ref{tab:performance}, the multi-file approach introduces minimal overhead while significantly improving error detection capabilities.

\subsection{User Experience Improvements}

The multi-file system provides several advantages over traditional single-file editing:

\begin{enumerate}
    \item \textbf{Improved Navigation}: Users can quickly jump between sections using the file tree
    \item \textbf{Parallel Editing}: Multiple authors can work on different sections simultaneously
    \item \textbf{Modular Development}: Sections can be developed and tested independently
    \item \textbf{Reusability}: Common sections (like methodology) can be shared across projects
\end{enumerate}

\subsection{Cross-Reference Validation}

The system successfully handles complex cross-referencing scenarios:
\begin{itemize}
    \item Equation references across files (e.g., Equation \ref{eq:quadratic} from mathematical-equations.tex)
    \item Table references (e.g., Table \ref{tab:performance} in this section)
    \item Section references spanning multiple files
    \item Bibliography citations integrated seamlessly \cite{knuth1984literate}
\end{itemize}

\subsection{Error Recovery}

Testing revealed robust error recovery capabilities:
\begin{itemize}
    \item Syntax errors in individual files don't break the entire compilation
    \item Missing file references are clearly reported with file paths
    \item Circular dependencies are detected and prevented
    \item Recovery suggestions are provided for common LaTeX errors
\end{itemize}

The results demonstrate that the multi-file approach not only maintains compilation reliability but actually improves it through better error isolation and reporting.