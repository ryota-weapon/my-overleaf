\section{Methodology}

This section describes the technical approach used to implement the multi-file LaTeX editing system.

\subsection{System Architecture}

The web-based LaTeX editor follows a client-server architecture with the following components:

\begin{enumerate}
    \item \textbf{Frontend}: React-based user interface with file tree navigation
    \item \textbf{Backend}: Node.js server with Express.js framework
    \item \textbf{Compilation Engine}: Integration with pdflatex and bibtex
    \item \textbf{File Management}: RESTful API for CRUD operations on project files
\end{enumerate}

\subsection{File Organization Strategy}

Projects are organized using a hierarchical structure where:
\begin{itemize}
    \item Each project has a unique identifier
    \item Files are stored in a directory tree structure
    \item The main.tex file serves as the entry point
    \item Section files are included using \texttt{\textbackslash input\{filename\}}
    \item Bibliography files use the .bib extension
\end{itemize}

\subsection{Compilation Process}

The compilation workflow involves multiple passes to ensure proper handling of cross-references and citations:

\begin{enumerate}
    \item Initial pdflatex run to process document structure
    \item BibTeX execution for bibliography processing (if .bib files present)
    \item Second pdflatex run to resolve citations
    \item Final pdflatex run to resolve all cross-references
\end{enumerate}

This multi-pass approach ensures that all references, both internal (equations, figures, tables) and external (citations) are properly resolved in the final PDF output.

\subsection{Error Handling}

The system implements comprehensive error handling:
\begin{itemize}
    \item LaTeX compilation errors are parsed and displayed to users
    \item File system errors are caught and reported appropriately
    \item Network failures are handled gracefully with retry mechanisms
    \item Syntax validation helps prevent common LaTeX mistakes
\end{itemize}